\documentclass{tufte-handout}

\title{An Introduction to Git}

\author[Dr. Doeg]{The Invisible College}

%\date{28 March 2010} % without \date command, current date is supplied

%\geometry{showframe} % display margins for debugging page layout

\usepackage{graphicx} % allow embedded images
  \setkeys{Gin}{width=\linewidth,totalheight=\textheight,keepaspectratio}
  \graphicspath{{graphics/}} % set of paths to search for images
\usepackage{amsmath}  % extended mathematics
\usepackage{booktabs} % book-quality tables
\usepackage{units}    % non-stacked fractions and better unit spacing
\usepackage{multicol} % multiple column layout facilities
\usepackage{lipsum}   % filler text
\usepackage{fancyvrb} % extended verbatim environments
  \fvset{fontsize=\normalsize}% default font size for fancy-verbatim environments
  
  
  
    %MADNESS
  
  \usepackage[T1]{fontenc} % Use 8-bit encoding that has 256 glyphs
\usepackage{fourier} % Use the Adobe Utopia font for the document - comment this line to return to the LaTeX default
\usepackage[english]{babel} % English language/hyphenation
\usepackage{amsmath,amsfonts,amsthm} % Math packages
\usepackage{mathtools}% http://ctan.org/pkg/mathtools
\usepackage{etoolbox}% http://ctan.org/pkg/etoolbox
\usepackage{lipsum} % Used for inserting dummy 'Lorem ipsum' text into the template
\usepackage{units}% To use \nicefrac
\usepackage{cancel}% To use \cancel
\usepackage{physymb}%To use r
\usepackage{sectsty} % Allows customizing section commands
\usepackage[dvipsnames]{xcolor}
\usepackage{pgf,tikz}%To draw 
\usepackage{pgfplots}%To draw 
\usetikzlibrary{shapes,arrows}%To draw 
\usetikzlibrary{patterns,fadings}
 \usetikzlibrary{decorations.pathreplacing}%To draw curly braces 
 \usetikzlibrary{snakes}%To draw 
 \usetikzlibrary{spy}%To do zoom-in
 \usepackage{setspace}%Set margins and such
 \usepackage{3dplot}%To draw in 3D
\usepackage{framed}%To get shade behind text


\definecolor{shadecolor}{rgb}{0.9,0.9,0.9}%setting shade color
\allsectionsfont{\centering \normalfont\scshape} % Make all sections centered, the default font and small caps
  
  

  
  

% Standardize command font styles and environments
\newcommand{\doccmd}[1]{\texttt{\textbackslash#1}}% command name -- adds backslash automatically
\newcommand{\docopt}[1]{\ensuremath{\langle}\textrm{\textit{#1}}\ensuremath{\rangle}}% optional command argument
\newcommand{\docarg}[1]{\textrm{\textit{#1}}}% (required) command argument
\newcommand{\docenv}[1]{\textsf{#1}}% environment name
\newcommand{\docpkg}[1]{\texttt{#1}}% package name
\newcommand{\doccls}[1]{\texttt{#1}}% document class name
\newcommand{\docclsopt}[1]{\texttt{#1}}% document class option name
\newenvironment{docspec}{\begin{quote}\noindent}{\end{quote}}% command specification environment

\begin{document}

\maketitle% this prints the handout title, author, and date
\begin{marginfigure}%
  \includegraphics[width=\linewidth]{obi.jpg}
  \caption{The git mascot is the Octocat.  This cowled Octocat is a Jedi master.}
  \label{fig:marginfig}
\end{marginfigure}
\begin{abstract}
\noindent
Git is a distributed revision control system with an emphasis on speed, data integrity, and support for distributed, non-linear workflows.  Git was initially designed and developed by Linus Torvalds for Linux kernel development in 2005, and has since become one of the most widely adopted version control systems for software development.
\end{abstract}

\vspace{1cm}
\normalsize
\section{Tell Git Who You Are}

\marginnote[40pt]{Tag your commits with name and email.  You must do this the first time use use git.}
\begin{shaded}
\begin{verbatim}
$ git config --global user.name "Dr Doeg"
$ git config --global user.email doeg@example.com
\end{verbatim}
\end{shaded}

\vspace{1cm}

\section{Basic Terminal Fu }

Using the terminal you may navigate the file directory.  Make, delete, move and rename files and directories. 

\begin{margintable}[130pt]\index{unixfu}
  \footnotesize%
  \begin{center}
    \begin{tabular}{ll}
      \toprule
     Unix Command & Action \\
      \midrule
     \bf{cd}  & change directory        \\
    \bf{ls}  & list files        \\
     \bf{cp}  & list files        \\
      \bf{mv}  & move files        \\
       \bf{rm}  & remove files        \\
        \bf{rmdir}  & remove directory        \\
         \bf{touch}  & create file        \\
          \bf{nano}  & edit file        \\
           \bf{mkdir}  & create directory        \\
      \bottomrule
    \end{tabular}
  \end{center}
  \caption{A list of Unix shell commands.}
  \label{tab:font-sizes}
\end{margintable}


\begin{shaded}
\begin{verbatim}
$ cd path/to/project/folder
$ ls
$ cp filename ~/Location/newname
$ mv filename ~/Location/newname
$ rm filename
$ rmdir directoryname
$ touch filename
$ mkdir directoryname
$ nano filename
\end{verbatim}
\end{shaded}


\section{Get Git to Tell You Where It's At}

\marginnote[25pt]{Get information on the git repository.  Do not initialize a new git every time you return to a file.  If there is already a git for the project folder than use it! }
\begin{shaded}
\begin{verbatim}
$ git status
\end{verbatim}
\end{shaded}

\vspace{1cm}

\section{Setting Up a Local Git Repository}

\marginnote[40pt]{Using the command line navigate to the project folder and initialize a git repository.}

\begin{shaded}
\begin{verbatim}
$ git init
\end{verbatim}
\end{shaded}

\marginnote[40pt]{Add files in the folder to the stage.\\ \ \\
\noindent Or add all the files.  }
\begin{shaded}
\begin{verbatim}
$ git add file2.jpg
$ git add .
\end{verbatim}
\end{shaded}

\marginnote[40pt]{Commit the additions.  }
\begin{shaded}
\begin{verbatim}
git commit -m "comment on the file changes"
\end{verbatim}
\end{shaded}

\vspace{1cm}

\section{Push Your Local Repository to GitHub}

\marginnote[25pt]{\normalsize \bf{Setup the remote repository location on GitHub using your account.}}
\begin{shaded}
\begin{verbatim}
$ git remote add origin https://github.com/<USER>/<REPO>.git
\end{verbatim}
\end{shaded}

\marginnote[40pt]{If you already set up the remote and want to change it use "set-url".}
\begin{shaded}
\begin{verbatim}
$ git remote set-url origin https://..../<USER>/<REPO>.git
\end{verbatim}
\end{shaded}

\marginnote[40pt]{Push the committed structure to the remote server.}
\begin{shaded}
\begin{verbatim}
$ git push origin master
\end{verbatim}
\end{shaded}


\newpage

\section{Cloning an Existing Repository From GitHub }


\marginnote[40pt]{Navigate to the desired location in file structure.}
\begin{shaded}
\begin{verbatim}
$ cd path/to/whereUwant/folder
\end{verbatim}
\end{shaded}

\marginnote[40pt]{Set the location on the GitHub server to place the repositiory.}
\begin{shaded}
\begin{verbatim}
$ git clone https://github.com/<USER>/<REPO>.git
\end{verbatim}
\end{shaded}

\vspace{1cm}

\section{Working With Branches }
Version control is one of the great powers of git.

\begin{margintable}[80pt]\index{branchfu}
  \footnotesize%
  \begin{center}
    \begin{tabular}{ll}
      \toprule
     Unix Command & Action \\
      \midrule
     \bf{branch}  & list branches       \\
    \bf{branch} <NAME>  & create new branch        \\
    \bf{checkout} <NAME>  & switch to new branch        \\
     \bf{merge} <NAME>  & merge branch with current      \\
      \bf{branch -m} <NAME>  & rename current branch \\
      \bf{branch -D} <NAME>  & delete branch       \\
      \bottomrule
    \end{tabular}
  \end{center}
  \caption{A list of git commands for version control.}
  \label{tab:font-sizes}
\end{margintable}

\begin{shaded}
\begin{verbatim}
$ git branch
$ git branch branchname
$ git checkout branchname
$ git merge branchname
$ git branch -m newbranchname
$ git branch -D branchname
\end{verbatim}
\end{shaded}

\vspace{1cm}

\section{Updating an Existing Repository From GitHub }

\marginnote[40pt]{The sophisticated way to update uses fetch, reviews changes and merges those onto the master branch.  The alt the current project folder from the GitHub remote server.}
\begin{shaded}
\begin{verbatim}
$ git fetch
$ git pull -u origin master
\end{verbatim}
\end{shaded}

\vspace{1cm}

\section{Get Git, Github and More on Git}

\marginnote[20pt]{Download git and register an account at GitHub. Look at the official documentation for more information.}
https://git-scm.com/downloads  \\
\noindent https://github.com/  \\
\noindent  https://git-scm.com/book/en/v2

\bibliography{sample-handout}
\bibliographystyle{plainnat}



\end{document}
